\documentclass{article}

\author{Matthew Coyle}
\title{Theorie des Langages}

\begin{document}
Exo 1\\
$R = (ba)^*$\\
baba, , ba, bababa, ...\\


$R=(b|a)^*$\\
aa, abbab, , babbbabaadbab, ...\\


$R=a(ba)^*b$\\
ab, abab, abababab, ... = $ \{(ab)^n , n \geq 1\} $\\


$R=(m|t)o(m|t)o$\\
momo, moto, toto, tomo\\


Exo 2\\
L = {mots se terminant par 1}\\
$ (0|1)^*.1 $\\

L = {mots contenant au moins un 1}\\
$ (0|1)^*.1.0^* $\\

L = {mots alternant 0 et 1 commencant par 0}\\
$ 0.(10)^*.(1|\varepsilon) $\\

L = {mots contenant un nombre pair 1}\\
$ (0^*.1.0^*.1)^*.0^* $\\

L = {mots contenant au moins une fois 11}\\
$(0|1)^*.11.(0|1)^*$\\

L = {mots contenant au moins un 0 et au moins un 1}\\
$ ((0|1)^*.0.(0|1)^*.1)^+ | ((0|1)^*.1.(0|1)^*.0)^+ $\\


Exo 3\\
a=1, b=1, c=0, d=1, e=0\\

Exo4\\
$ (1|\varepsilon)(00^*1)^*0^* $ represente le langage qui commance par au plus un 1, qui repete le motif: au moins un 0 suivi d'un 1, et qui fini par un nombre quelconque de 0.\\


Exo5\\
(aa)*\\ 
(aa)+\\
(aa)*\\
a(aa)*\\
a*b\\
(aa)*a\\

Exo6\\



\end{document}