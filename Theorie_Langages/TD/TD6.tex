1
1.1	
	ltr= 34
	rtl= 24
1.2 
	O(nm) |motif|=n, |text|=m

2
2.1
	sur feuille
2.2
	O(n) n=longueur de text
2.3 
	i	Mi 			a 		b
	0	\varepsilon	a 		\varepsilon
	1 	a 			a 		ab
	2 	ab 			aba 	\varepsilon
	3 	aba 		a 		abab
	4 	abab 		aba 	ababb
	5 	ababb 		a 		\varepsilon

	O(n^3|\sigma|)
2.4
	M = ababb
	M_0 = \varepsilon
	M_1 = a  		plbord(1) = \varepsilon
	M_2 = ab 		plbord(2) = \varepsilon
	M_3 = aba 		plbord(3) = a
	M_4 = abab 		plbord(4) = ab
	M_5 = ababb 	plbord(5) = \varepsilon

2.5, 2.6
	sur feuille

2.7
	



(Parenthese)
Calcule de S(i,x)
Si i =0 et M[i]!= x Alors
	retourner 0
Si i=m Alors
	i == plbord(i)
Tant que i > 0 et M[i]!=x
	i == plbord(i)
Si M[i]=x
	retourner i+1
sinon retourner 0



3
M = bacaba
T = bdacbacbacababadaca

3.1
3.1.a) d = {1,3,3,(1),1,(1),6} 	(17 cmp)
3.1.b) d = {4,1,2,4,2,2} (16cmp)
3.1.c) d = {4,3,4,5} (12cmp)

3.2
der_occ = [3|4|2|-1]
decl_bs = [4|4|4|4|4|2|1]
		  -1 0 1 2

3.3