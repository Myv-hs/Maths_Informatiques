\documentclass{article}

\author{Matthew Coyle}
\title{Theorie des Langages}

\begin{document}

remaques:
i ly a u et un seul setat initial
il peut y avoir 0,1 ou plusieurs etats acceptants
les transitions p-a->q portent sur des symboles a de $\sum$ pas des chaines $\omega$ de $\sum$
fonction transition totale
sugma q a est tjrs defnini

P puis reste tjrs dans P

\end{document}