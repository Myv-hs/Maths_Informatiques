\documentclass{article}

\author{Matthew Coyle}
\title{Theorie des Langages}

\begin{document}
\maketitle

\section{Introduction}

Les langages formels on ete etudies par:\\
- les informaticiens $=>$ langages de programmation:\\
(definir syntace, verifier la syntace d'un programme, le traduire en langage machine)\\
- les linguistes $=>$ langues naturelles\\
(les decrire et essayer de les traiter automatiquement)\\
\\
Exemples de langages:\\ 
- Les entiers naturels (suites de chiggres parmis 0..9)\\
- Les entiers naturels impairs (meme representation)\\
- Les mots francais (du dictionaire)\\
- Les phrases en francais\\
- Les programmes (syntaxiquement corrects) ecrtis en C++\\
\\
Points communs:\\
- Chaque langages est un ensemble d'elements appeles mots ou "chaines".\\
- Chaque chaine est une suite de symboles pris parmi un ensemble fini de symboles.\\
- Chaque chaine est de longeur finie.\\
\\
On etudie des modeles pour representer de maniere finie des langages:\\
automate finis\\
expressions refulieres\\
grammaire formelles\\

Application pratiques:\\
Recherche de "motifs" dans les fichiers\\
traitement de texte\\
modelisation de circuits\\
modelisation de machines a etats\\
Compliation de langages de programmation\\
\\

un alphabet est un ensemble fini, non vide, de symoles. On le note generalement $\sum$.\\
Example d'alphabets:\\
$\sum_{entiers}$ = \{0,1,2,3,4,5,6,7,8,9\}\\
$\sum_{mots}$ = \{a,b,c,...,z,',-\}\\
$\sum_{ident}$ = \{a,...,z,A,...,Z,0,...,9,\_\}\\
$\sum_{prog}$ = \{int, float, bool, while, <, toto, a, ...\}

U mot ou une chaine w forme(e) sur un alphabet est une suite finie $s_1 s_2 ... s_n$ de symboles de cet alphabet\\


La concatenation de deux chaines u et v, notee u.v ou uv est la chaine obtenue en ecrivant les symboles de u suivis de ceux de v.\\
si $u=a_1 a_2 ... a_n$ et $v=b_1 b_2 b_p$\\
alors $uv=a_1 a_2 ... a_n b_1 b_2 ... b_p$\\

Un prefixe d'une chaine w est une suite de symboles debutant w.\\
Un suffixe de w est une suite de symboles terminant w.\\
$\forall x,y \mid w=x.y$, x est un prefixe de w, y un suffixe.\\

Une sous-chaine d'une chaine w est une suite de symboles apparaissant consecutivement dans w.\\

Un langage est un ensemble de chaine.

Example de langages:\\
\{toto,titi,tata\}\\
\{1,11,101,1001\}\\
$\{1^n \mid n \geq 0\} = \{e,1,11,111,1111,11111,...\}$\\
Nombres binaires impaires: \{1,11,101,111,1001,1011,...\}\\
Nombres binaires premiers: \{1,10,11,101,111,1011,...\}\\
\\
le Langage vide, note $\o$, ne contient aucune chaine (ensemble vide).\\

Remaque: $\o \ne \{\varepsilon\}$\\
Le langage plein, note $\sum^*$\\

l'Union de deux langages A et B est le langage, note $A \cup B$, compose de toutes les chaines qui apparaissent dans l'un au moins de langages A ou B.\\

LA concatenation de deux langages A et B est le langage, note A.B ou AB, compose de toutes les chaines formees par une chaine de A concatenee a une chaine de B.\\
$A.B = \{u.v \mid u \in A$ ou $ v \in B\}$\\

Proprietes:\\
Associativite: (A.B).C = A.(B.C) \\
$\{\varepsilon\}$ est un element neutre: $A.\{\varepsilon\} = \{\varepsilon\}.A = A$\\
$\o est element absorbant$\\
Distributivite de la concatenation sur l'union:\\

Puissance d'une langage A
$A^k$ est le langage forme par la concatenation de k occurrences de A.\\
$A^0 = \{\varepsilon\}$\\
$A^1 = A$\\
\\

Etoile de Kleene (fermeture ou cloture par *).\\
- la fermeture de Kleene d'un langage A est le langage, note $A^*$\\
$A^* = \bigcup|_{i=0}^\infty A^i$\\



\section{Modeles et Langages}
- contextuels\\
- langages recursivement enumerables\\
\end{document}