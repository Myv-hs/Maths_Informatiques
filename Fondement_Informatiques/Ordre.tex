\documentclass{article}

\author{me}

\begin{document}

Def Une relation d'odre large est une relation reflexive, anti-seymetrique, transitive.
Une relation d'ordre strict est une relation irreflexive, anti-symetrique, transitive.

Si R est une relation d'ordre strict sur E, alors la relation R U IDe est une relation d'ordre large sur E.

Reciproquement, si R' est une ordre large sur E, alors R' \ IDe est une relation d'odre strict sur E

notation ordre large \leq stric <

on passe d'une ordre large a un ordre strict par les equivalences.

x < y \iff x \leq y et x != y
x \leq y \iff x<y et x=y

Ordres totaux, partiels.
Si R relation d'ordre verifie \forall e, e' \in E, e!=e' \rightarrow (eRe') ou (e'Re), on dit que R est im ordre total. Sinon R est un ordre partiel.

EX:

Ordre naturel sur les Reels est un ordre total.

La divisibilite sur les entiers naturels.
	a \leq_div b ssi \exists c t.q. b = ac

	6 \leq_div 12, 6 !\leq_div 15

L'inclusion sur P(E) est un ordre partiel


Pre-ordre: Une relation de preordre est une relation transitive.

Ensemble Ordonne, Un ensembke ordonne (E, \leq) est une ensemble muni d'une relation d'ordre.

Application monotones.

Def: Soient (E_1, S_1) et (E_2, S_2) deux ensembles ordonnes.
Une application f de E_1 dans E_2 est dite monotonne ssi:
\forall x,y \in E_1, x\leq_1 y \rightarrow f(x) \leq_2 f(y)
On parle d'homomorphisme de l'enemble ordonne (E_1, \leq_1) dans l'ensemble ordonne (E_2, \leq_2).
On dira que (E_1, \leq_1) et (E_2, \leq_2) sont isomorphes s'il existe une bijectionb entre E_1 et E_2 t.1. b et b^-1 soient monotones.

Ex: si deux ensembles ordonnes (E_1, S_1) et (E_2, S_1) on le meme support, E_1 == E_2
l'inclusion de \leq_1 dans \leq_2, \leq_1 \subseteq \leq_2, c-a-d \forall x,y x\leq_1 y \rightarrow x \leq_2 y
reviens a dire que l'application Id de (E_1, \leq_1) dans (E_2, \leq_2) est monotonne.

Remarque: Il ne suffit pas qu'une bijection soit monotonne pour qu'elle soit un isomorphisme. 
par exemple l'application Id de (\N, \leq_div) dans (\N,\leq) est une bijection monotonne mais par un isomorphisme (la reciproque n'est pas monotonne).


Ensembles totalement ordonnes

Un ensemble ordonne (E, \leq) est totalement ordonne si \leq est une ordre total.
i.e. ssi \forall x,y \in E, x!=y \rightarrow x\leqy or y\leqx
Il est partiellement ordonne si \exists x,y, \in E t.q. x!=y, x!\leqy et y!\leqx

Soit (E, \leq) un ensemble partiellement ordonne, une extention lineaire de (E,\leq) est une ensemble totalement ordonne (E,\leq_t) de meme support et t.q. \leq \subseteq \leq_t

(Tri topologique)

Theoreme:
Soit (E, \leq) un ensemble ordonne, il a aumoins une extention lineaire et \leq est egale a l'intersection de toutes ces extentions lineaires.
Ex(fig)

Algo: On choisit un element sans predecesseur, on l'enleve et on le place en tete de liste, et on recommance.

Produit d'ensemble ordonne:
Soient (E_1, \leq_1), (E_2,\leq_2), le produit direct de ces deux ensembles ordonnes est (E_1 x E_2, \leq) avec la relation \leq definie par: (x_1, x_2) \leq (y_1,y_2) ssi x_1 \leq_1 y_1 et x_2 \leq_2 y_2
Remarque: L'ordre sur le produit direct est aussi appele ordre produit.

Le produit lexicographique de (E_1, \leq_1) et (E_2, \leq_2) est (E_1 x E_2, \leq) avec (x_1, x_2) \leq (y_1, y_2) ssi (x_1 \leq y_1 ou x_1 = y_1 et x_2 \leq y_2)

sous-ensembles ordonnes, chaines, anti-chaines

Soit (E,\leq) un ensemble ordonne,  Un sous-ensemble ordonne, (E', \leq') de (E,\leq) est t.q. E' \subseteq E et \leq' = \leq \cap (E' x E')
c-a-d \forall x,y \in E' x\leq'y sii x \leq y
Une chaine de E est un sous-ensemble totalement ordonne de E.
Une chaine est maximale si elle n'est pas strictement incluse dans une autre chaine.
Une anti-chaine E' de E est un sous-ensemble de E t.q.: \leq \cap (E' x E') = Id_{E'} <(-.-)

Une anti-chaine est maximale si elle n'est pas strictement incluse dans une autre anti-chaine


Majorants, minorants.
Def: Soir E' une partie d'un ensemble ordonne (E,\leq) un element x de E est un majorant (resp minorant) de E' si \forall y \in E', y \leq x (resp x \leq y)

Prop: Maj(E') \cap E' et Min(E') \cap (E') ont chacun au plus un element.
Si Maj(E') \cap E' != \o , l'unique element de cet ensemble est appele le maximum de E' (le plus grand element de E')
Si Min(E') \cap E' != \o , l'unique element de cet ensemble est appele le minimum de E' (le plus petit element de E')


Prop: Soit E' une partie de E et z \in E, les 3 conditions suviantes sont equivalents.
(1) z est le max de E'
(2) z \in E' et \forall x \in E' x \leq z
(3) z \in E' et z est le minimum de Maj(E')

Soit E' une partie de E. Un element x de E' est dit maximal dans E' si \forall y \in E' Y >= x \rightarrow y=x. Si E' a un element maximum, alors c'est son unique maximal.

Def. Un element x \in E est la borne superieur d'une partie E' de l'ensemble E ordonne
ssi: (\forall y \in E', y \leq x) et (\forall z \in E, ((\forall y \in E', y \leq z)\rightarrow x \leq z))

x est le plus petit des majorants de E' = brone superieur de E'

Def: Un element x \in E est la borne inferieur d'une partie E' de l'ensemble E ordonne
ssi: (\forall y \in E', x \leq y ) et (\forall z \in E, ((\forall y \in E', z \leq y)\rightarrow z \leq x))
x est le plus grand des minorants de E' la borne inf de E'

On note sup(E') et inf(E') les bornes superieurs et inferieurs de E' si elles existent

Prop soit E' \subseteq E
(1) Si z est le max de E', alors z = sup(E')
(2) Si sup(E') \in E, alors sup(E') est le maximum de E'

Ex \N ordonne par la divisibilite, (\N, \leq_{div})
Pour cet ordre, la borne inferieure d'un ensemble de 2 entiers existe toujours et elle est le PGCD
Pour cet ordre, la borne superieur d'un ensemble de 2 entiers existe toujours et elle est le PPCM (plus petit commun multiple)

Rem: les bornes superieurs et inferieurs n'existent pas toujours.
Ex E = \{a,b,c,d\} ordonne par \{a \leq c, a \leq d, b \leq c, b \leq d \}
\{a,b\} n'as ni bornes superieurs ni bornes inferieurs, pareil pour \{c,d\}

Prop: Soit {E_i}_{i\in I}, une famille de parties d'un ensemble ordonne (E, \leq) e E' = \cup_{i\in I} E_i
Si chaque ensemble E_i admet une borne superieur (resp. inferieur) e_i
et si l'ensemble \{e_i, i \in I\} admet une borne superieur (resp. inferieur) e

alors e est ka borne superieur (resp. inferieur) de E'



Ensemble bien fonde et Induction
Def: Une relation d'ordre \leq sur un ensemble E est bien fondee s'il n'y a pas de suite infinie strictement decroissante sur E.
Un bon ordre est un ordre total bien fonde.

Un ensemble ordonne (E, \leq) est bien fonde ssi toute partie non vide de E admet au moins un element mimnimal.

Principe d'induction sur les ensemblse bien fondes
th: Soit \leq un ordre bien fonde sur E et P une proposition sur un element x de E
si la propriete (I) suivant est verifiee (I) \forall x \in E, ((\forall y < x, P(y))\rightarrow P(x))

alors \forall x \in E, P(x)



Treillis
Def: Un ensemble ordonne (E, \leq) est un Treillis si toute paire d'elements admet une borne superieur et une borne inferieur
On note x \sqcup y au lieu de sup(\{x,y\})
on note x \sqcap y au lieu de inf(\{x,y\})

Ex: (\N, \leq_{div}) est un treillis.
les operation binaires 
\sqcup ppcm
\sqcap pgcd
\bot 1 \top 0


Si E est un Treillis on peut considerer que E est muni de 2 operations binaires \sqcup et \sqcap


Proprietes de \sqcap \sqcup
- deupotence x \sqcup x = x et x \sqcap x = x
- commutativite x \sqcup y = y \sqcup x et x \sqcap y = y \sqcap x
- associativite (x\sqcup y)\sqcup z = x \sqcup (y\sqcup z) et (x\sqcap y)\sqcap z = x \sqcap (y\sqcap z)
- absoption x \sqcap (x\sqcup y) = x et x \sqcup (x\sqcap y) = x

Reciproquement, si sur un ensemble E, 2 operations binaires ont es 4 proprietes precedents.
On peut ordonner E de facon a ce que x \sqcup y et x \sqcap y soient respectivement la borne superieur et la borne inferieur de x et y
Il suffit de poser x \leq y ssi x\sqcup y =y, ce qui a cause de la propriete d'absorption est equivalent a x \sqcap y = x

x \sqcup y = y \rightarrow x \sqcap (x \sqcup y) = x \sqcap y \rightarrow x = x \sqcap y

Toute partie finie non-vide d'un treillis a une borne superieur et une borne inferieur


Prop: les deux operations \sqcap et \sqcup sont monotones.
i.e.: 	si x \leq x' et y \leq y'
		alors x \sqcup u \leq x' \sqcup y'
			et x \sqcap u \leq x' \sqcap y'

Def un Terillis est distributif si \sqcap et \sqcup distribuent l'une par rapport a l'autre.
i.e.: si 
1) \forall x,y,z x \sqcup(y \sqcap z) = (x \sqcup y) \sqcap (x \sqcup z)
2) \forall x,y,z x \sqcap(y \sqcup z) = (x \sqcap y) \sqcup (x \sqcap z)
Prop: 1) et 2) sont equivalents

Def: Un treilis est dit complemente si
1) il a un elements minimum \bot et un element maximum \top distincts (\bot \not= \top)
2) \exists une application \mu de E dans E (\mu.E -> E) t.q.:
\forall x \in E x \sqcap \mu(x) = \bot
\forall x \in E x \sqcup \mu(x) = \top

Ex: le treillis P(E) est complemente
1) \bot = \o \top = E
2) l'application \mu est l'operation habituelle de complement

\forall A \in P(E), A \subseteq E, A \sqcap \mu(A) = \o
A \sqcup \mu(A) = E
A \sqcap \overline{A} = \o
A \sqcup \overline{A} = E


Proposition: Si un Treillis complemente est distributif il n'existe qu'une seule operation de complementation \mu
Cette operation posede en plus les prop suivantes.
Elle est involuptible \forall x \mu(\mu(x))=x
Elle verifie les lois de Morgan:
\forall x,y \mu(x \sqcup y) = \mu(x) \sqcap \mu(y)
		et  \mu(x \sqcap y) = \mu(x) \sqcup \mu(y)
x \leq y \leftrightarrow \mu(y) \leq \mu(x)

Ex: le treillis E = \{a,b,c,\bot,\top\}
fig 

Il a (au moins) 2 operations de complemtation \mu et \Mu
par ex: \mu(\top)=\bot, \mu(\bot)=\top
\mu(a)=b, \mu(b)=c, \mu(c)=a
\Mu(\top)=\bot, \Mu(\bot)=\top, \mu(a)=c, \mu(b)=a, \mu(c)=b


Treillis complet et fonctions continues:

Def: Un ensemble ordonne (E, \leq) est appele un treillis complet si toute partie de E admet une borne superieur et une borne inferieur

Ex: (P(E), \subseteq) est un treillis complet. 
Sup({E_i}_{i \in I}) = \cup_{i \in I} E_i.
Inf({E_i}_{i \in I}) = \cap_{i \in I} E_i.

Si E est un treillis complet, la borne inferieure de E est majoree par tous les elements de E = un treillis complet comporte donc un element minimum note \bot
\bot = Inf(E)
idem pour born sup: \top
Sup(E) = \top

Si E est un treillis complet, \o a une borne sup et comme Maj(\o)=E, Sup(\o) = \bot
On a donc Inf(E) = Sup(\o)=\bot
et Inf(\o) = Sup(E) = \top

Def : Une application f d'un ensemble ordonne (E_1, \leq_1) dans un ensemble ordonne (E_2, \leq_2) est dite continue (sup_continue) si elle preserve les bornes superieurs des parties non vides.
Si la partie E' \not= \o a une borne superieur e = sup(E') alors f(E') = \{f(x) t.q. x \in E'\} a aussi une borne superieur qui est egale a f(e)
Sup(f(E'))=f(sup(E'))

Rem: Puisque dans un treillis complet, les bornes sup existent toujours, la continuite d'une application s'exprime alors par:
	f(sup(E))=sup(f(E))

Prop: tout fonction continue est monotone
soient x,y \in E_1 t.q. x \leq y
alors sup(\{x,y\}) = y et la continuite de f implique que sup(\{f(x),f(y)\}) = f(sup(\{x,y\}))=f(y), et donc f(x) \leq f(y)


\end{document}