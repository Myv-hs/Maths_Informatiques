\documentclass{article}

\author{Matthew Coyle}
\title{Fondement de Mathematiques Informatique}

\begin{document}

\maketitle


\section{Calculs ensemblistes, fonctions}

\subsection{ensemble, elements, inclusion:}


Soit E un ensemble, et e un element, $e \in E$ signifie que e est un element qui est dans E, et se lit e appartient a E.
La negation de cette relation, e n'appartiens pas a E se note $e \notin E$.\\
$\o$ est l'ensemble qui ne contiens auccun element.\\

A et B, deux ensemble. On dit que A est un sous-sensemble de B ou une partie de B, ou encore que A est inclus dans B et on note $A \subseteq B$\\
sois $\forall x \in A => x \in B$\\

A = B ssi $A \subseteq B$ et $B \subseteq A$ , A et B on les memes elements.\\
La negation de $A \subseteq B$ s'ecrit $A \not\subseteq B$\\

On note $\mathcal{P}(E)$, l'ensemble des Parties de E ( l'ensemble des sous-ensembles de E, note $2^E$ )

car $card(\mathcal{P}(E)) = 2^{card(E)}$

$E = \{1,3,8,12\}$ $card(\mathcal{P}(E))=2^4$

$\mathcal{P}(E)=\{\o,\{1\},\{3\},\{8\},\{12\},\{1,3\},\{1,8\}, ... , \{1,3,8,12\}\}$\\

A $\subseteq E$ ssi $A \in \mathcal{P}(E)$


Remarque: 
	$\o \in \mathcal{P}(E), E \in \mathcal{P}(E) \forall E$\\


Produit Cartesien de 2 ensembles E et F. C'est l'ensemble des couples formes d'un element de E et d'un element de F.\\
	$E*F = \{(x,y) \mid x \in E$ et $y \in F\}$\\
\clearpage
Generalisation:\\
On generalise le produit cartesien a une famille finie d'ensembles:\\
$E_1 * E_2 * ... * E_n = \{(x_1,x_2,...,x_n), x1 \in E_1, x_2 \in E_2, ... , x_n \in E_n\}$

$E^n = E * E * E ....$ n fois $n \geq 1$

$E^n$ peut etre defini recusivement
par 

$$E^1 = E$$

$$E^n = E * E^{n-1}$$\\


Remarque:\\
	D'un point de vue strictement formel, le produit des ensembles n'est pas associatif.\\
$(E*F)* G \ne E * (F*G)$\\
$((x,y),z) \in (E*F)*G$ et $(x,(y,z)) \in E*(F*G)$\\
mais on peut etablir une bijection entre les 2.
D'une facon generale on n'as pas $E^n * E^m = E^{n+m}$



\subsection{Reunion, Intersection, difference complementaire, partition}

Referentiel E\\
Soient A et B deux parties de E, on definit 
l'intersection de A et B: $A \cap B = \{e \in E \mid e \in A $ et $ e\in B\}$\\
l'union de A et B: $A \cup B = \{e \in E \mid e \in A$ ou $e\in B\}$\\
la differance de A et B: $A - B = \{e\in E \mid e\in A$ et $e\notin B\}$\\
le complementaire de A dans E: $\overline{A}$ ou $A^c  = E\mid \overline{A} = \{e\in E \mid e \notin \overline{A}\}$\\
la difference symetique de A et B $A \triangle B = (A\setminus B)\cup (B\setminus A)$\\
On dit que A et B sont disjoints ssi $A\cap B = \o$

\end{document}