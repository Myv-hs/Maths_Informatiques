\documentclass{article}

\author{me}
\newtheorem{thm}{Theorem}

\begin{document}
$
Def: Soit (E, \leq) un ensemble ordonne, a \leq on associe une relation pred (precedence) par x pred y \leftrightarrow x \leq y et \notexists z t.q. x \leq z \leq y avec x \not=y,x\not=z,z\not=y

Rem: La relation pred peut etre triviale. Par exemple sur les rationnels ordonnes par l'ordre naturel, la relation pred est vide.

Th ordre \rho sur un ensemble fini E est uniquement determine par la relation de precedence associes.
Plus precisemment \rho = pred* (fermeture, transitive, reflexive de pred)

Calcul:
On suppose que l'on represente une relation d'ordre R par une matrice binaire t.q. 

R(i,j) = \begin{cases} 1 si iRj \\ 0 sinon \end{cases}

On va se poser le pbe de trouver les sucesseurs immediats des elements de E. cad Comment trouver la matrice H de la relation R t.q. H definie par xHy ssi y est un successeur immediat de x

\begin{thm}Si S designe le carre de R vue comme une matrice positive a coef \in \N, on a:

H(i,j)=\begin{cases} 1 si S(i,j) = 2 \\ 0 sinon \end{cases}

On appelle 1,2,3,...,n les elements de E
on a S = R^2 = R.R
S_{i,j} = R_{1i}R_{1j} + R_{2i}R_{2j}+ ... + R_{ni}R_{nj}

S_{i,j} = 2 \rightarrow j successeur immediat de i.

\end{thm}

Mathode pour construire le diagramme de Hasse.
1) Determiner R la matricec de la relation d'ordre
2) Calcucler S = R^2 en considerant R comme une matrice positive.
3) A chaque fois que S(i,j)=2 dessiner une fleche de i a j.


Ex: sur (\N_5, R)
R = \{1\leq1,1\leq2,1\leq3,1\leq4,1\leq5,2\leq2,2\leq3,2\leq4,2\leq5,3\leq3,4\leq4,5\leq5\}

Matrice 1a
\begin{tabular}{ c c c c c c}
 &1&2&3&4&5 \\ 
1&1&1&1&1&1 \\  
2&0&1&1&1&1 \\
3&0&0&1&0&0 \\
4&0&0&0&1&0 \\
5&0&0&0&0&1
\end{tabular}

Matrice 1b
\begin{tabular}{ c c c c c c}
 &1&2&3&4&5 \\ 
1&1&1&1&1&1 \\  
2&0&1&1&1&1 \\
3&0&0&1&0&0 \\
4&0&0&0&1&0 \\
5&0&0&0&0&1
\end{tabular}

Matrice 2a
\begin{tabular}{ c c c c c c}
S&1&2&3&4&5 \\ 
1&1&2&3&3&3 \\  
2&0&1&2&2&2 \\
3&0&0&1&0&0 \\
4&0&0&0&1&0 \\
5&0&0&0&0&1
\end{tabular}

Matrice 2b
\begin{tabular}{ c c c c c c}
&1&2&3&4&5 \\ 
1&0&1&0&0&0 \\  
2&0&0&1&1&1 \\
3&0&0&0&0&0 \\
4&0&0&0&0&0 \\
5&0&0&0&0&0
\end{tabular}


$
\end{document}