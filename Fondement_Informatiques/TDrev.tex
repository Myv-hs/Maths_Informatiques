9.2.1
Avec 4 couleurs, on peux tirer maximum 4 cartes de couleurs differentes. Un de plus (5 cartes) garentissent que 2 sont de meme couleur. Similairement, 8 cartes et le maximum de cartes tiré avec 2 cartes de chaque couleur, une de plus obligerais a avoir 3 carte de meme couleur.

9.2.2
Avent de tirer 3 coeurs on pourrais tirer tout les treffles, piques, et carreaux, sois 39 cartes, avent de tirer le premier coeur. Pour s'assurer d'avoir 3 coeurs on dois tirer 42 cartes.



9.3
E=\{0,1\}
E^2= \{(0,0),(0,1),(1,0),(1,1)\}

= 	(0,0),(1,1) symetrique, reflexive, transitive
!=	(0,1),(1,0) symetrique, reflexive
<	(0,1) transitif
>	(1,0) transitif
!<	(1,0) transitif
!>	(0,1) transitif
<=	(0,0),(0,1),(1,1) anti-symetrique, reflexive, transitive
>=	(0,0),(1,0),(1,1) anti-symetrique, reflexive, transitive


9.4


9.5




10.1.1
(0,1) , (2,1) , (6,1), (6,7), (2,3), (4,1)

10.1.2

(B) (0,1) \in S 
	a=0 pair, b=1 impair

(I) (a,b) \in S \rightarrow (a+2,b) \in S 
	a+2 et b conserve la parite' de a et b

(I) (a,b) \in S \rightarrow (2(a+b),b) \in
	2(a+b) force a devenir pair

(I) (a,b) \in S \rightarrow (a,ab+b)
	si a est pair et b est impair ab+b est impair,
	c'est le cas de a et b dans le cas de base,
	et toutes les inductions concervent cette parite'
	alors celle ci la concervera aussi

	a est toujours pair 2n, b est toujours impair 2n+1 \rightarrow \forall(a,b) \in S (a+b) est impair

10.2
\{(a,a),(b,b),(c,c),(d,d),(c,a),(a,d),(b,c),(b,d),(b,a)\}
Reflexive
Trichotomique
Bien-fonde'


10.3
E=\{a,b,c,d,e,f,g\}

10.3.1

b-f
 \
  d
   \
  c-g
 /
a-e

10.3.2

10.3.3
\o


10.5
|E| = n
|P(E)| = 2^n

|E \setminus A| = n-p
|P(E \setminus A)| = 2^{n-p} le nombres de parties de E sans aucun element de A

Il y a p singleton dans A.

Toutes les parties de E qui ne continnent qu'un seul element de P et le resultat du produit cartesiens, de tout les parties de (E-A) et de tout les elements de A.
Et il y en a p2^{n-p}