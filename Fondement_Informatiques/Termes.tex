\documentclass{article}
\begin{document}

Il s'agit d'un exemple tres utile en informatique de definition par induction structurelle.

Definition:
soit F = \{f_0,...,f_n\} un ensemble de symboles d'operations. A chaque symbole est assosie' une arite' fini a(f) \in \N qui repreente le nombre de ses arguments.

Soit U l'ensemble de tout les suites de symboles appartenant a F\cup \{(,),,\}.
Soit F, l'ensemble des symboles d'arite' i (les symboles d'arite' 0 sont les constantes).

Definition: L'ensemble T des terms construit sur F est difini inductivement par:
(B) B=F_0 (symboles des constantes)
(I) \forall f \in F_n, d'arite' n, \phi(t_1,...,t_n) = f(t_1,...,t_n) \in T
    \forall t_1,...,t_n \in T

Un terme peut etre represente' sous forme d'un arbre et f(t_1,...,t_n) est alors represente' par


Interpretation des termes:
Soit V  un sensemble quelconque, A chaque element de f de F_0 on associe un element h(f) de V
A chaque element f de F_i (avec i > 0) on associe une application h_\phi : V' -> V
(h_\phi(v_1,v_2,...v_i) = v, v_1,...,v_i \in V et v \in V)

Prposition: Il existe une et une seue application de h* de T dans V tell que:
(B') si t \in F_0, h*(t) = h(t)
(I') si t = f(t_1, t_2,...,t_n), h*(t)=h_f(h*(t_1),...,h*(t_n))

Si t est un terme, l'element h*(t) de V sera appele' l'interpretation de t par h*

Example: Soit F_0 = \{a\} , F_1=\{s\}, F=F_0 \cup F_1
T = \{a,s(a),s(s(a))\}

Interpretation basse sur la structure des termes
Soit E un ensemble quelconque et c \subseteq E difini inductivement par les conditions (B) et (I)
Le theoreme indiquant que tout element de C peut s'obtenir a partir de la base en applicant un nombre fini d'etapes inductives peut etre affine' en decrivant par un terme la facon dont un element x est obtenu.
A chaque element b de B on associe un smbole \overline{b} d'arite' 0
A chaque fonction \phi on associe le symbole de fonction \overline{\phi} d'arite' a(\phi)

Soit T l'ensemble des termes construit avec ces ensembles de symboles. On considere l'interpretation h* T -> E defini par
-h(\overline{b})=b
-h_\phi (x_1,x_2,...,x_{a(\phi)}) = \phi(x_1,x_2,...,x_{a(\phi)})

Prposition X = \{h*(t) | t \in T\}

Definitions non ambigues:
Definition: Une definition inductive d'un ensemble x est dite non ambigue si l'application h* est injective c-a-d que \forall x \in X, \exists! t \in T, x = h*(t)
Tout element de X est l'interpretation d'un et un seul terme.
Plus intuitivement, cela signifie qu'il n'existe qu'une seul facon de construire un element x de X.

Example: la difinition suivant des \N^2 est ambigue:
(B) (0,0) \in \N^2
(I_1) (n,m) \in \N^2 \rightarrow (n+1,m) \in \N^2 f
(I_2) (n,m) \in \N^2 \rightarrow (n,m+1) \in \N^2 q

En effet le couple (1,1) peut etre obtenu de la base an applicant I_1 puis I_2, ou l'inverse.


Plus formelement, on considere les termes formes a partir:
- du symbole \overline{b} d'arite' 0 dont l'interpretation est h(\overline{b}) = (0,0)
- des symboles unaires \overline{f} et \overline{q} dont les interpretations soit donnes par 
h_{\overline{f}}(n,m) = (n+1,m), h_{\overline{q}}(n,m) = (n,m+1),

On a (1,1) = h*(\overline{f}(\overline{q}(\overline{b}))) = h*(\overline{q}(\overline{f}(\overline{b})))

\end{document}