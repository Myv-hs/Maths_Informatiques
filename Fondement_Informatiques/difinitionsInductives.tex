\documentclass{article}
\begin{document}

Ensembles definis inducivement (recursivement).

De facon intuitive, la definition inductive d'une partie X d'un ensemble consiste en la donnee de certains elements de X, et en des moyens de construire des nouveaux elements dans X a' partir d'elements deja connus.

Une telle difinition presentatera donc de maniere generique:
	(B) certains elements de X donnees explicitements (base de la recursivite)
	(I) les autres elements de X soit definis en fonction d'elements appartenant deja a X (etape inductive de la definition recursive)

	0 \in \N_even
	si a \in \N_even alors a+2 \in \N_even

	X est le plus petit ensemble vefifiant (B) et (I).

	De facon plus formelle:
Definition: Soit E un ensemble. Une definition infuctive d'une partie X de E consiste en la donnee:
	-d'un sous-ensemble B de E
	-d'un sensemblel K d'operation \phi. E^{a(\phi)} -> E ou' a(\phi) est l'arite de \phi

	(B) \subseteq X
	(I) \forall \phi \in K, \forall x_1, x_2,...,x_{a(\phi)} \in X, \phi(x_1, x_2,...,x_{a(\phi)}) \in X

L'ensemble defini X = \cap Y (t.q. Y \in F) ou F = \{ y \subseteq E, B \subseteq Y et Y verifie (I)\}

Notons qu'en general plusieurs ensembles verifient les coritions (B) et (I)

Ex:
(B) 0 \in P
(I) \forall n \in \N, n \in P \rightarrow n+2 \in P

Il existe une inifite d'ensembles verifiant (B) et (I)

\N, \N\setminus\{1\}, \N\setminus\{1,3\},  \N\setminus\{1,3,5\}, ... sont de tels sous-ensembles (ou parties de \N). La partie P definie par (B) et (I) n'est aucune de celles-la' puisqu'il s'agit de l'ensemble des entriers pairs.

On notera aussi:
(B) \forall x \in B, x \in X
(I) \forall \phi \in K, x_1, x_2,...,x_{a(\phi)} \in E \rightarrow \phi(x_1, x_2,...,x_{a(\phi)}) \in X.

Ex: la partie X de \N definie inductivement par 
	(B) 0 \in X
	(I) n \in X \rightarrow n+1 \in X.
C'est \N


Soit A=\{(,)\}. L'ensemble D \subseteq A^* des parenthesages bien formes apple langage de Dyck est defini inductivement par.
	(B) \varepsilon \in D
	(I) u,v \in D \rightarrow (u) \in D et uv \in D

L'ensemble E des expressions entierement parenthesees formees a partir d'identificateurs apparenant a N et des operateurs + et * et de \{(,)\}
est la partie de (N \cup \{+,*,(,)\})^*
definie inductivement par:
	(B) N \subseteq E
	(I) e,f \in E (e+f) \in E, (e*f) \in E


L'ensemble AB des arbres binaires se definit inductivement par c'est la partie de 
	(A \cup \{\o,<,>,,\})^* ensemble d'etiquette
definie inductivement par
	(B) \o \in AB
	(I) l,r \in AB, a \in A \rightarrow <a,g,d> \in AB


Theoreme: Si X est definipar les conditions (B) et (I), tout element de X peut s'obtenir a' partir de la base en appliquent un nombre fini d'etapes infuctives.

Preuve par induction structurelle:
Le principe d'induction est une generalisation du principe de recurrence sur les entiers concue pour demontrer les proprieres des ensembles definis inductivement.
La preuve papr induction calque exactement la definition inductive de l'ensemble, c'est pourquoion l'appelle preuve par induction structurelle.


Proposition, Soit X un ensemble defini inductivement et soit P(x) un predicat enoncant une propriete sur un element x de X 
Si les conditions suivants sont verifiees.

(B") P(x) est vraie pour chaque x de B
(I") P(x_1,X_2,...,x_{a(\phi)}) \rightarrow P(\phi(x_1,x_2,...,x_{a(\phi)}))
\forall \phi \in k
alors P(x) est vraie \forall x de X


\end{document}