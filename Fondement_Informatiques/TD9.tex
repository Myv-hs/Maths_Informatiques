Exo1
F_0 = \{a,b\}, F_1=\{r\}, F_2=\{s\} et F = F_0 \cup F_1 \cup F_2

F = \{a,b,r(a),r(b),s(a,b),s(r(a),r(b)), s(s(a,b),s(a,b)), s(b,a), s(r(a),b), ...\}

Exo2
2.1
(B) x=\varepsilon, g(\varepsilon, y) = y \rightarrow Q(x) vrais pour la base.
x \in L

2.1 cor
Q(x): "\forall y, g(x,y) est defini" x,y \in L
Montrons par induction sur x que Q(x) vrai \forall x \in L

(B) \varepsilon \in L, Q(\varepsilon): "\forall y, g(\varepsilon,y) est defini" est vrais car g(\varepsilon, y) = y defini.
(I) \forall l \in L, Q(l) vrais, g(l) est defini \forall y \in L
Soit a \in L, g((al),y)=g(l,(ay)) \rightarrow defini
Donc \forall x \in L, \forall y \in L, g(x,y) defini

2.2
g((a_1),y) = a_1y

2.3
n=1 ok
n >= 2 :
Supposons que pour n > 1 l'egalite vraie
g((a_n(a_{n-1}(...(a_1)...))),y) = g(\varepsilon, (a_1(...(a_{n-1}(a_n))...))
Montrons que c'est vrai pour n+1
g((a_{n+1}(a_n(a_{n-1}(...(a_1)...)))),y) = g((a_n(a_{n-1}(...(a_1)...))),a_{n+1}y)
										  = g(\varepsilon, (a_1(...(a_{n-1}(a_n))...)a_{n+1}y)

2.4
rev(x) = g(x,\varepsilon)
rev((a_n(a_{n-1}(...(a_1)...)))) = g((a_n(a_{n-1}(...(a_1)...))),\varepsilon)
								 = (\varepsilon, (a_1(...(a_{n-1}(a_n))...)a_{n+1}y)
								 = a_1(a_2(...(a_n)...))


Exo3
3.1
(B) \forall x,y x<y x\%y = x
(I) \forall x,y x>y x\%y = (x-y)\%y


3.1 brouillon
modulo(n,m)
	if(n>=m) return modulo(n-m,m)
	return n

modulo(n,m)
	if(n<m) return n
	return modulo(n-m,m)

3.2
\N x \N* coincide avec la partie X de \N x \N defini par
(B) \forall x \in \N, \forall y \in \N* x<y, (x,y) \in X
(I) (x,y) \in X \rightarrow (x+y,y) \in X