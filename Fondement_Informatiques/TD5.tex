P(n) "9|10^n-1"
Q(n) "9|10^n+1"

Montrer que \forall n \in \N: 
	P(n) \rightarrow P(n+1)
	Q(n) \rightarrow Q(n+1)

Soit n \in \N, supposons P(n)
\Exists k t.q. k9 = 10^n - 1

Montrons P(n+1):
10^{n+1} - 1 	= 10^{n+1}-10+10-1
				= 10(10^n-1})+10-1
				= 90k+9
				= 9(10k+1)
Donc P(n) \rightarrow P(n+1)


Soit n \in \N, supposons Q(n)
\Exists k t.q. k9 = 10^n + 1

Montrons Q(n+1):
10^{n+1} + 1	= 10(10^n +1) - 10 +1
				= 9(10k-1)

Soit n \in \N, R(n) vraie. \Exists k\in\N, 10^n+1 = 9k+2

Montrons R(n+1):
10^{n+1} + 1	= 10(10^n+1)-10 +1
				= 90k+11 = 9(10k+1)+2


Demontrer par reacurrence que: 
\forall n \in \N, \Existsm \in \N
tel que \sum^n_{k=0} k^3 = m^2

On va montrer que 
\sum^n_{k=0} = (n(n+1)/2)^2= (1/4)(n(n+1))^2

n=0 vrais
sOit n \in \N, supposons que \sum^n_{k=0} k^3 = 1/4 (n(n+1))^2
Montrons que c'est vrai pour n+1
\sum^{n+1}_{k=0} k^3 = \sum^n_{k=0} k^3 + (n+1)^3
					= (n+1)^3 + (1/4)(n(n+1))^2
					= (n+1)^2 ((n+1)+(1/4)n^2)
					= (n+1)^2 ((1/4)(n+2)^2)
					= (1/4)((n+1)(n+2))^2