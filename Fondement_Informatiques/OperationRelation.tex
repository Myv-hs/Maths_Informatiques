\documentclass{article}

\author{Matthew Coyle}
\title{Fondement de Mathematiques Informatique}

\begin{document}
Une Operation $\phi$ sur un ensemble E, est une application $\phi$: $E^n -> E$.
$\phi$ est d'arite n, le rang de $\phi$ est n, $\phi$ est une operation n-aire.
$a(\phi)=n$

Operation binaires. Une operation binaire est une loi de composition interne *. Sur un ensemble E est une application $*:ExE=>E$
l'image d'un couple (x,y) par * est note n*y.

Props:
* est associative ssi $\forall a,b,c \in E$ a*(b*c) = (a*b)*c
* est commutatuve ssi $\forall a,b \in E$ a*b = b*a
* admet un element neutre 1 ssi $\forall e \in E$ e*1=1*e=e

Def: Un ensemble muni d'une operation * associative est un semi-group. De plus si E posede un element neutre e pour *, alors (E,*,e) est un monoide. Si * est commutative, le semi-groupe (resp. le monoide) est commutatif.

Exemple: (P(E),$\cap$) est un monoid commutatif
(P(E),$\cup$) est un monoid commutatif

Soit A un ensemble fini appele alphabet et dont les elements sont appelees lettres, le monoide libre sur A note $A^*$ est l'ensemble des mots ecrits sur A.
Un mot u: suite fini de lettres.
|u|:longeur du mot.
|$\varepsilon$| le mot vide.

La loi de composition: la concatenation de deux mots:
$u=u_1u_2...u_n$ et $v=v_1v_2...v_n$ $u.v=u_1u_2...u_nv_1v_2...v_n$

Remarque: $\ne$ entre les suites et les mots d'une langage.
les elements d'un mot appartiennent a un ensemble fini, alors que les elements d'une suite appartienne a un ensemble infini (peuvent.)
Les mots on toujours une longeur finie pas les suites.
$\varepsilon$ mot qui ne contient aucun element.

Difinition recursive d'une mot:
Soit $\sum$ un alphabet, $\omega \in \sum^*$ si:
	$\omega = \varepsilon$
	$\omega = x.u, x \in \sum et u \in \sum^*$

$E_1 = {u \in {a,b}^* | |u|=5 }$

Un ensemble E muni d'une operation * est un group si c'est un monoide et que tout elements admet un inverse:
$\forall e \in E, \exists e' \in E | e*e'=e'*e=1$

Si * est commutative, le group est commutatif

Ex: \Z muni de l'addition est un group commutatif.





Relations.

Def: Une relation sur un ensemble E est la donne d'une partie de R de ExE.
Une paire (e,e') de ExE est dans R. eRe', (e,e')$\in$ R, R(e,e')

Ex: ${(n,m)|n \seq m}$
${(n,m)|n \seq m \seq 2n}$

Operation ensemblists sur les relations.

le complementaire R d'une relation R dans $E^2$

$(e,e') \in R <=> (e,e') \in R_1 ou (e,e') \in R_2$


l'Union:
$(e,e') \in R_1 \cup R_2 <=> (e,e') \in R_1 ou (e,e') \in R_2 $

l'Union:
$(e,e') \in R_1 \cap R_2 <=> (e,e') \in R_1 et (e,e') \in R_2 $

la relation vide:
$ \forall e,e' \in E, (e,e') \not\in \o_E $

la relation pleine:
$ \forall e,e' \in E, (e,e') \in \pi_E $

la relation vide:
$ \forall e,e' \in E, (e,e') \in Id_E <=> e = e' $


Relation binaire sur E,
relation inverse:
$ eR^-1e' <=> e'Re $

Produit de deux relation binaires $R_1R_2$
$ e(R_1.R_2)e' <=> \exists e'' | (eR_1e'') et (e''R_2e') $
produit assiciatif et a $Id_E$ comme element neutre.

$ R^* = Id_E \cup R \cup (R.R) \cup (R.R.R) \cup ... $

$ \cup_{i\geq0} avec R^0 = Id_E$
$R^{1+1} = R.R^i$


$R^+ = \cup_{i>0} R'$

donc $R^* = ID_E \cup R^+$
$ \forall i, j \geq 0 R^{i+j}=R^i.R^j $


//////////YASMINA\\\\\\\\\\\\


Relation d'equivalance

Une relation d'equivalance est une relation reflective symetrique, transitive.
L'egalite sur un ensemble E est une relation d'equivalance.
L'intersection $R \cap R'$ de 2 relations d'equivalance est une relation d'quivalence. Mais pas necessairement $ R \cup R' $ ni R.R'

Def: R relation d'quivalence sur E. e element de E

$ {e' \in E | eRe'} = [e]_R est la classe d'equivalancec de e. $

Prop.
$ \forall e \in E, e \in [e]_R $
$ \forall e \in E, eRe' => [e]_R = [e']_R$
$ [e]_R \cap [e']_R \ne \o => [e]_R= [e']_R $

$ {[e]_R | e \in E} $ ensemble de partie de E est appele ensemble quotient de R par R, E/R est une partition de E.
Reciproquement.


Congruence.

Def Une relation d'equivalance R definie sur un ensemble E muni d'une loi de composition interne * est une congruence.
Si elle est compatible avec la loi *
c.a.d si:
$ \forall e, e', d, d' \in E (eRe') et (dRd') \rightarrow ((e*d)R(e'*d'))$

Si R est une congruence sur E muni de *, la loi passe au quotient c'est a dire que E/R est muni d'une loi [*]
en posant e[*]e'=[e*e'] et [*] est bien definie (ne depens pas de representants choisis).


Prop: Soit R une congruence sur un monoide (resp group)(E,*)

E/R muni de la loi [*] est un monoide (resp group).


\end{document}